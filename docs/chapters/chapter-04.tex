\chapter{Interrogazione dell'Ontologia}

\section{SWRL}\label{sec:swrl}
In questa sezione, esamineremo quattro regole SWRL implementate nell'ontologia carpedia.
Queste regole sono state create per derivare nuove informazioni o classificare istanze in base a determinate
condizioni o vincoli specifici all'interno dell'ontologia.
Ognuna di queste regole svolge un ruolo specifico nel processo di arricchimento delle conoscenze nell'ontologia.

\subsection{3DoorsBody}

\begin{lstlisting}[language=Prolog]
carpedia:Body(?body) ^
carpedia:hasMechanicalComponent(?body, ?door1) ^
carpedia:hasMechanicalComponent(?body, ?door2) ^
carpedia:hasMechanicalComponent(?body, ?door3) ^
differentFrom(?door1, ?door2) ^
differentFrom(?door2, ?door3) ^
differentFrom(?door1, ?door3) ->
autogen0:DoorsBody(?body)
\end{lstlisting}


Questa regola e' stata creata per classificare una struttura del corpo dell'automobile come "DoorsBody" quando
soddisfa le seguenti condizioni: ha esattamente 3 componenti meccanici distinti che rappresentano porte e ognuna
di queste porte e' diversa dalle altre due.
L'obiettivo e' quello di identificare correttamente il tipo di carrozzeria delle automobili con una
specifica configurazione di porte, che potrebbe avere implicazioni su aspetti come la sicurezza o l'aerodinamica.

\subsection{5DoorsBody}

\begin{lstlisting}[language=Prolog]
carpedia:Body(?body) ^
carpedia:hasMechanicalComponent(?body, ?door1) ^
carpedia:hasMechanicalComponent(?body, ?door2) ^
carpedia:hasMechanicalComponent(?body, ?door3) ^
carpedia:hasMechanicalComponent(?body, ?door4) ^
carpedia:hasMechanicalComponent(?body, ?door5) ^
differentFrom(?door1, ?door2) ^
differentFrom(?door2, ?door3) ^
differentFrom(?door1, ?door3) ^
differentFrom(?door1, ?door4) ^
differentFrom(?door1, ?door5) ^
differentFrom(?door2, ?door4) ^
differentFrom(?door2, ?door5) ^
differentFrom(?door3, ?door4) ^
differentFrom(?door3, ?door5) ^
differentFrom(?door4, ?door5) ->
autogen1:DoorsBody(?body)
\end{lstlisting}


Questa regola e simile a "3DoorsBody", ma e' stata creata per classificare una struttura del corpo dell'automobile
come "DoorsBody" quando ha esattamente 5 componenti meccanici distinti che rappresentano porte.
Questa regola puo' essere utile per distinguere tra automobili con configurazioni di porte diverse,
come ad esempio quelle con 3 o 5 porte, influenzando il design e l'usabilita'.

\subsection{NewlyLicensedCar}

\begin{lstlisting}[language=Prolog]
carpedia:Car(?c) ^
carpedia:hasEngine(?c, ?e) ^
carpedia:hasKilowatts(?e, ?k) ^
swrlb:lessThan(?k, 70) ^
carpedia:hasWeight(?c, ?w) ^
swrlb:divide(?x, ?k, ?w) ^
swrlb:lessThan(?x, 0.055) ->
carpedia:NewlyLicensedCar(?c)
\end{lstlisting}


Questa regola e' stata creata per classificare un'automobile come "NewlyLicensedCar" quando soddisfa
determinati criteri, tra cui una potenza dell'unita' motrice inferiore a 70 kW e un rapporto potenza/peso
inferiore a 0,055.
Questo potrebbe essere utile per identificare le automobili che potrebbero essere di interesse per i conducenti
neo patentati o per chi e' alla ricerca di veicoli con prestazioni piu' moderate.

\subsection{Supercar}

\begin{lstlisting}[language=Prolog]
carpedia:Car(?c) ^
carpedia:hasMaxSpeedKmph(?c, ?m) ^
swrlb:greaterThan(?m, 290) ^
carpedia:hasEngine(?c, ?e) ^
carpedia:hasKilowatts(?e, ?k) ^
swrlb:greaterThan(?k, 297) ->
carpedia:Supercar(?c)
\end{lstlisting}


Questa regola e' stata creata per classificare un'automobile come "Supercar" quando soddisfa condizioni
specifiche, tra cui una velocita' massima superiore a 290 km/h e una potenza dell'unita' motrice superiore a 297 kW.
Questa regola puo' essere utile per identificare le automobili ad alte prestazioni all'interno dell'ontologia,
distinguendole dalle automobili tradizionali.

\subsection{Considerazioni}
Queste regole SWRL sono state sviluppate per arricchire l'ontologia delle automobili con informazioni
aggiuntive e per consentire la classificazione o la derivazione di nuove conoscenze in base a criteri specifici.
Ogni regola ha uno scopo specifico e contribuisce alla comprensione delle automobili all'interno del contesto
dell'ontologia.


\section{SPARQL}\label{sec:sparql}

In questa sezione, esamineremo tre query SPARQL implementate nell'ontologia carpedia,
sviluppate per ottenere informazioni significative sulla base di criteri specifici.
Queste interrogazioni sono state create per esplorare diverse caratteristiche delle automobili all'interno
dell'ontologia e recuperare dati rilevanti in base a determinati requisiti.

\subsection{Automobili adatte alla guida nei centri urbani}

\begin{lstlisting}[language=SPARQL]
PREFIX rdf: <http://www.w3.org/1999/02/22-rdf-syntax-ns#>
PREFIX carpedia: <http://www.semanticweb.org/filippo/ontologies/2023/8/carpedia#>

SELECT ?car
WHERE {
  ?car rdf:type carpedia:Car .
  ?car carpedia:hasEuroEmissionClass ?euroEmissionClass .
  FILTER (?euroEmissionClass IN (carpedia:Euro4, carpedia:Euro5, carpedia:Euro6)) .
}
\end{lstlisting}


Questa query e' stata creata per identificare le automobili che sono adatte a circolare nei centri urbani,
dove spesso vengono introdotte restrizioni sulle emissioni.
Abbiamo selezionato tutte le automobili (?car) che sono istanze della classe carpedia:Car e che hanno una
classe di emissione Euro 4, 5 o 6 (?euroEmissionClass).
Questo ci permette di ottenere un elenco di automobili conformi alle normative Euro emission, che sono generalmente
considerate idonee per la guida in aree urbane.

\subsection{Automobili non adatte alla guida nei centri urbani}

\begin{lstlisting}[language=SPARQL]
PREFIX rdf: <http://www.w3.org/1999/02/22-rdf-syntax-ns#>
PREFIX carpedia: <http://www.semanticweb.org/filippo/ontologies/2023/8/carpedia#>

SELECT ?car
WHERE {
  ?car rdf:type carpedia:Car .
  ?car carpedia:hasEuroEmissionClass ?euroEmissionClass .
  FILTER (?euroEmissionClass IN (carpedia:Euro0, carpedia:Euro1, carpedia:Euro2, carpedia:Euro3)) .
}
\end{lstlisting}

Questa query e' stata progettata per individuare le automobili che non sono adatte alla guida nei centri urbani,
poiche' le loro emissioni possono superare i limiti consentiti.
Abbiamo selezionato tutte le automobili (?car) che sono istanze della classe carpedia:Car e che hanno una
classe di emissione Euro 0, 1, 2 o 3 (?euroEmissionClass).
Queste classi di emissione rappresentano automobili con standard di emissione meno stringenti,
che potrebbero non essere autorizzate nelle zone urbane piu' pulite.

\subsection{Automobili adatte alle famiglie}

\begin{lstlisting}[language=SPARQL]
PREFIX rdf: <http://www.w3.org/1999/02/22-rdf-syntax-ns#>
PREFIX carpedia: <http://www.semanticweb.org/filippo/ontologies/2023/8/carpedia#>

SELECT ?car
WHERE {
  ?car rdf:type carpedia:Car .
  ?car carpedia:hasSeats ?seats .
  FILTER (?seats >= 5) .
}
\end{lstlisting}


Questa query e' stata formulata per identificare le automobili che sono adatte alle famiglie, in base al numero
di posti a sedere. Abbiamo selezionato tutte le automobili (?car) che sono istanze della classe carpedia:Car e
che hanno almeno 5 posti a sedere (?seats >= 5). Questo ci permette di individuare le automobili che possono
ospitare comodamente famiglie numerose o gruppi di persone.

\subsection{Automobili a Benzina}

\begin{lstlisting}[language=SPARQL]
PREFIX rdf: <http://www.w3.org/1999/02/22-rdf-syntax-ns#>
PREFIX carpedia: <http://www.semanticweb.org/filippo/ontologies/2023/8/carpedia#>

SELECT ?car
WHERE {
  ?car rdf:type carpedia:Car .
  ?car carpedia:hasSystem ?system .
  ?system rdf:type carpedia:FuelSupplySystem .
  ?system carpedia:hasFuel ?fuel .
  ?fuel rdf:type carpedia:Gasoline
}
\end{lstlisting}


Questa query e' stata progettata per identificare tutte le automobili che utilizzano la benzina come carburante.
Estrae le istanze della classe carpedia:Car che hanno un sistema di alimentazione (carpedia:hasSystem) associato,
dove il tipo di carburante (carpedia:Gasoline) di quel sistema e' benzina.
Questa query puo' essere utile per ottenere un elenco di automobili che utilizzano specificamente la benzina
come carburante.

\subsection{Automobili che costano meno di 20000 euro}

\begin{lstlisting}[language=SPARQL]
PREFIX rdf: <http://www.w3.org/1999/02/22-rdf-syntax-ns#>
PREFIX carpedia: <http://www.semanticweb.org/filippo/ontologies/2023/8/carpedia#>

SELECT ?car
WHERE {
  ?car rdf:type carpedia:Car .
  ?car carpedia:hasPrice ?price .
  FILTER(?price < 20000)
}
\end{lstlisting}

Questa query e' stata sviluppata per individuare le automobili che hanno un prezzo inferiore a una certa
soglia (in questo caso, 20000). Estrae le istanze della classe carpedia:Car che hanno un attributo di prezzo
(carpedia:hasPrice) inferiore alla soglia specificata. Puo' essere utilizzata per filtrare automobili in base
al prezzo, consentendo agli utenti di trovare veicoli che rientrano nel loro budget.

\subsection{Automobili con meno di 100000 Chilometri percorsi}

\begin{lstlisting}[language=SPARQL]
PREFIX rdf: <http://www.w3.org/1999/02/22-rdf-syntax-ns#>
PREFIX carpedia: <http://www.semanticweb.org/filippo/ontologies/2023/8/carpedia#>

SELECT ?car
WHERE {
  ?car rdf:type carpedia:Car .
  ?car carpedia:hasKilometers ?kilometers .
  FILTER(?kilometers < 100000)
}
\end{lstlisting}


Questa query mira a individuare le automobili che hanno percorso meno di un certo numero di chilometri
(in questo caso, 100000). Estrae le istanze della classe carpedia:Car che hanno un attributo di chilometraggio
(carpedia:hasKilometers) inferiore alla soglia specificata.
Puo' essere utilizzata per trovare automobili con una bassa usura o con un minor chilometraggio.

\subsection{Automobili prodotte da Audi}

\begin{lstlisting}[language=SPARQL]
PREFIX rdf: <http://www.w3.org/1999/02/22-rdf-syntax-ns#>
PREFIX carpedia: <http://www.semanticweb.org/filippo/ontologies/2023/8/carpedia#>

SELECT ?car
WHERE {
  ?car rdf:type carpedia:Car .
  ?car carpedia:hasManufacturer ?manufacturer .
  FILTER(?manufacturer = carpedia:Audi)
}
\end{lstlisting}


Questa query e' stata progettata per identificare le automobili prodotte da un produttore specifico
(in questo caso, carpedia:Audi). Estrae le istanze della classe carpedia:Car che hanno un attributo di produttore
(carpedia:hasManufacturer) uguale al produttore specificato. Puo' essere utilizzata per ottenere un elenco
di automobili prodotte da un marchio specifico.

\subsection{Query 8: Automobili prodotte dopo il 2000}

\begin{lstlisting}[language=SPARQL]
PREFIX rdf: <http://www.w3.org/1999/02/22-rdf-syntax-ns#>
PREFIX carpedia: <http://www.semanticweb.org/filippo/ontologies/2023/8/carpedia#>

SELECT ?car
WHERE {
  ?car rdf:type carpedia:Car .
  ?car carpedia:hasProductionYear ?year .
  FILTER(?year > 2000)
}
\end{lstlisting}


Questa query estrae le automobili prodotte dopo un certo anno (in questo caso, il 2000).
Estrae le istanze della classe carpedia:Car che hanno un attributo di anno di produzione (carpedia:hasProductionYear)
superiore all'anno specificato.
Puo' essere utilizzata per ottenere un elenco di automobili prodotte nel XXI secolo.


\subsection{Considerazioni}
Queste query SPARQL consentono di ottenere informazioni significative e rilevanti dalle ontologie delle automobili,
offrendo un'ampia gamma di informazioni utili per diversi scopi, come la ricerca di automobili adatte all'ambiente
urbano o alle esigenze delle famiglie.
