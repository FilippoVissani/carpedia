\chapter{Interrogazione dell'Ontologia}

\section{SWRL}\label{sec:swrl}
Sono state utilizzate regole SWRL per consentire al reasoner di inferire ulteriori informazioni sull'ontologia.

\section{SPARQL}\label{sec:sparql}

In questa sezione, esamineremo tre query SPARQL implementate nell'ontologia delle automobili,
sviluppata per ottenere informazioni significative sulla base di criteri specifici.
Queste interrogazioni sono state create per esplorare diverse caratteristiche delle automobili all'interno
dell'ontologia e recuperare dati rilevanti in base a determinati requisiti.

\subsection{Query 1: Automobili adatte alla guida nei centri urbani}

Mettere foto o codice query 1

Questa query è stata creata per identificare le automobili che sono adatte a circolare nei centri urbani,
dove spesso vengono introdotte restrizioni sulle emissioni.
Abbiamo selezionato tutte le automobili (?car) che sono istanze della classe carpedia:Car e che hanno una
classe di emissione Euro 4, 5 o 6 (?euroEmissionClass).
Questo ci permette di ottenere un elenco di automobili conformi alle normative Euro emission, che sono generalmente
considerate idonee per la guida in aree urbane.

\subsection{Query 2: Automobili non adatte alla guida nei centri urbani}

Mettere foto o codice query 2

Questa query è stata progettata per individuare le automobili che non sono adatte alla guida nei centri urbani,
poiché le loro emissioni possono superare i limiti consentiti.
Abbiamo selezionato tutte le automobili (?car) che sono istanze della classe carpedia:Car e che hanno una
classe di emissione Euro 0, 1, 2 o 3 (?euroEmissionClass).
Queste classi di emissione rappresentano automobili con standard di emissione meno stringenti,
che potrebbero non essere autorizzate nelle zone urbane più pulite.

\subsection{Query 3: Automobili adatte alle famiglie}

Mettere foto o codice query 2

Questa query è stata formulata per identificare le automobili che sono adatte alle famiglie, in base al numero
di posti a sedere. Abbiamo selezionato tutte le automobili (?car) che sono istanze della classe carpedia:Car e
che hanno almeno 5 posti a sedere (?seats >= 5). Questo ci permette di individuare le automobili che possono
ospitare comodamente famiglie numerose o gruppi di persone.

\subsection{Considerazioni}
Queste query SPARQL consentono di ottenere informazioni significative e rilevanti dalle ontologie delle automobili,
offrendo un'ampia gamma di informazioni utili per diversi scopi, come la ricerca di automobili adatte all'ambiente
urbano o alle esigenze delle famiglie.
