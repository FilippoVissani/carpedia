\chapter{Interrogazione dell'Ontologia}

\section{SWRL}\label{sec:swrl}
In questa sezione, esamineremo quattro regole SWRL implementate nell'ontologia carpedia.
Queste regole sono state create per derivare nuove informazioni o classificare istanze in base a determinate
condizioni o vincoli specifici all'interno dell'ontologia.
Ognuna di queste regole svolge un ruolo specifico nel processo di arricchimento delle conoscenze nell'ontologia.

\subsection{Regola 1: 3DoorsBody}

Mettere foto o codice regola 1

Questa regola e' stata creata per classificare una struttura del corpo dell'automobile come "DoorsBody" quando
soddisfa le seguenti condizioni: ha esattamente 3 componenti meccanici distinti che rappresentano porte e ognuna
di queste porte e' diversa dalle altre due.
L'obiettivo e' quello di identificare correttamente il tipo di carrozzeria delle automobili con una
specifica configurazione di porte, che potrebbe avere implicazioni su aspetti come la sicurezza o l'aerodinamica.

\subsection{Regola 2: 5DoorsBody}

Mettere foto o codice regola 2

Questa regola e simile a "3DoorsBody", ma e' stata creata per classificare una struttura del corpo dell'automobile
come "DoorsBody" quando ha esattamente 5 componenti meccanici distinti che rappresentano porte.
Questa regola può essere utile per distinguere tra automobili con configurazioni di porte diverse,
come ad esempio quelle con 3 o 5 porte, influenzando il design e l'usabilita'.

\subsection{Regola 3: NewlyLicensedCar}

Mettere foto o codice regola 3

Questa regola e' stata creata per classificare un'automobile come "NewlyLicensedCar" quando soddisfa
determinati criteri, tra cui una potenza dell'unita' motrice inferiore a 70 kW e un rapporto potenza/peso
inferiore a 0,055.
Questo potrebbe essere utile per identificare le automobili che potrebbero essere di interesse per i conducenti
neo patentati o per chi e' alla ricerca di veicoli con prestazioni piu' moderate.

\subsection{Regola 4: Supercar}

Mettere foto o codice regola 3

Questa regola e' stata creata per classificare un'automobile come "Supercar" quando soddisfa condizioni
specifiche, tra cui una velocita' massima superiore a 290 km/h e una potenza dell'unita' motrice superiore a 297 kW.
Questa regola puo' essere utile per identificare le automobili ad alte prestazioni all'interno dell'ontologia,
distinguendole dalle automobili tradizionali.

\subsection{Considerazioni}
Queste regole SWRL sono state sviluppate per arricchire l'ontologia delle automobili con informazioni
aggiuntive e per consentire la classificazione o la derivazione di nuove conoscenze in base a criteri specifici.
Ogni regola ha uno scopo specifico e contribuisce alla comprensione delle automobili all'interno del contesto
dell'ontologia.


\section{SPARQL}\label{sec:sparql}

In questa sezione, esamineremo tre query SPARQL implementate nell'ontologia carpedia,
sviluppate per ottenere informazioni significative sulla base di criteri specifici.
Queste interrogazioni sono state create per esplorare diverse caratteristiche delle automobili all'interno
dell'ontologia e recuperare dati rilevanti in base a determinati requisiti.

\subsection{Query 1: Automobili adatte alla guida nei centri urbani}

Mettere foto o codice query 1

Questa query e' stata creata per identificare le automobili che sono adatte a circolare nei centri urbani,
dove spesso vengono introdotte restrizioni sulle emissioni.
Abbiamo selezionato tutte le automobili (?car) che sono istanze della classe carpedia:Car e che hanno una
classe di emissione Euro 4, 5 o 6 (?euroEmissionClass).
Questo ci permette di ottenere un elenco di automobili conformi alle normative Euro emission, che sono generalmente
considerate idonee per la guida in aree urbane.

\subsection{Query 2: Automobili non adatte alla guida nei centri urbani}

Mettere foto o codice query 2

Questa query e' stata progettata per individuare le automobili che non sono adatte alla guida nei centri urbani,
poiche' le loro emissioni possono superare i limiti consentiti.
Abbiamo selezionato tutte le automobili (?car) che sono istanze della classe carpedia:Car e che hanno una
classe di emissione Euro 0, 1, 2 o 3 (?euroEmissionClass).
Queste classi di emissione rappresentano automobili con standard di emissione meno stringenti,
che potrebbero non essere autorizzate nelle zone urbane piu' pulite.

\subsection{Query 3: Automobili adatte alle famiglie}

Mettere foto o codice query 2

Questa query e' stata formulata per identificare le automobili che sono adatte alle famiglie, in base al numero
di posti a sedere. Abbiamo selezionato tutte le automobili (?car) che sono istanze della classe carpedia:Car e
che hanno almeno 5 posti a sedere (?seats >= 5). Questo ci permette di individuare le automobili che possono
ospitare comodamente famiglie numerose o gruppi di persone.

\subsection{Considerazioni}
Queste query SPARQL consentono di ottenere informazioni significative e rilevanti dalle ontologie delle automobili,
offrendo un'ampia gamma di informazioni utili per diversi scopi, come la ricerca di automobili adatte all'ambiente
urbano o alle esigenze delle famiglie.
