\chapter{Analisi del Dominio}

Il macro dominio in cui si colloca Carpedia è rappresentato dal settore dell'automobile. In particolare, per "automobile" si intende un veicolo a motore con quattro ruote, progettato per il trasporto di persone e destinato all'uso privato dei cittadini. L'analisi sarà pertanto circoscritta escludendo i concetti relativi ai mezzi di trasporto pubblico, commerciali o destinati al lavoro. All'interno di questo vasto contesto, l'ontologia si concentra specificamente sui concetti correlati alle componenti di un'automobile, siano esse di natura sensoristica o meccanica, nonché sui sistemi che queste componenti compongono.

Tra i concetti inclusi nell'analisi rientrano le varie parti sensoristiche, meccaniche ed elettroniche che costituiscono l'architettura di un veicolo. Ciò comprende, ad esempio, sensori di vario genere, sistemi di gestione e controllo del motore, unità di trasmissione, sistemi di illuminazione, impianti di raffreddamento e componenti elettriche. L'ontologia si propone di offrire una rappresentazione dettagliata di ciascuna di queste componenti, consentendo una comprensione approfondita delle relazioni e delle interazioni che caratterizzano il funzionamento complessivo di un'automobile.

Inoltre, sarà considerata l'inclusione di concetti specifici legati alle emissioni, con particolare attenzione alle classi di emissioni Euro, al fine di fornire un quadro esaustivo della sostenibilità ambientale di un veicolo. La focalizzazione sui concetti strettamente legati alle automobili permette di creare un'ontologia mirata e specializzata, facilitando la gestione e l'interpretazione di dati nel contesto automobilistico. Questo approccio specifico è pensato per soddisfare le esigenze di chiunque abbia interesse a esplorare, analizzare o contribuire al vasto campo dell'ingegneria automobilistica, dalla progettazione dei veicoli alla manutenzione e alla gestione del ciclo di vita.

\section{Analisi dei Requisiti}
Di un'automobile si vogliono modellare i principali sistemi che la compongono, come ad esempio:

\begin{itemize}
    \item sistema frenante;
    \item motore di un'automobile: di questo sistema si vogliono modellare anche cilindrata e potenza;
    \item sistema di raffreddamento del motore;
    \item sistema di sterzo;
    \item sistema di gestione della batteria;
    \item sistema di gestione dell'olio motore;
    \item sistema di scarico;
    \item sistema di gestione del carburante;
    \item sistema di accensione;
    \item sistema di avviamento;
    \item sistema di alimentazione elettrica;
    \item carrozzeria;
    \item luci;
    \item sospensioni;
    \item trasmissioni;
\end{itemize}

Ogni sistema è composto da una serie di componenti, che possono essere di tre tipi:
\begin{itemize}
    \item sensore;
    \item indicatore: vale a dire un'interfaccia human-readable di una particolare misura rilevata da un sensore;
    \item componente meccanico di un sistema dell'automobile;
\end{itemize}

Dei componenti si vuole modellare anche il produttore ed il numero di serie.

Inoltre, si vuole anche modellare:

\begin{itemize}
    \item produttore di un'automobile o di un componente;
    \item il tipo di carburante utilizzato da un'automobile;
    \item la classe di emissioni di un'automobile;
\end{itemize}

Infine, si vogliono modellare ulteriori elementi dell'automobile, come ad esempio:

\begin{itemize}
    \item chilometri percorsi;
    \item velocità massima dell'automobile espressa in chilometri orari;
    \item prezzo;
    \item numero di posti;
    \item peso;
    \item modello dell'automobile;
    \item anno di produzione;
\end{itemize}