\chapter{Panoramica del Dominio}
In questa sezione verrà riportata una panoramica del dominio di Carpedia. In seguito, verrà descritta nel dettaglio la modellazione strutturale dell'ontologia, che esprime gli elementi di dominio attraverso i concetti
appartenenti al web semantico, come \textit{Classi}, \textit{Object Properties} e \textit{Data Properties}.

\section{Il dominio di Carpedia}
Il macro dominio in cui si colloca l'ontologia Carpedia è quello dell'\textit{automobile}. In particolare, con automobile si intende un veicolo a motore, con quattro ruote, progettato per il trasporto di persone
e destinato ad uso privato del cittadino. Saranno pertanto esclusi dall'analisi i concetti relativi a mezzi di trasporto pubblico, commerciali o di lavoro.\\

In questo enorme contesto, l'ontologia si focalizza sui concetti prettamente relativi alle \textit{componenti} di un'automobile, siano esse sensoristiche o meccaniche, e dei \textit{sistemi} che
esse compongono.\\

Un'automobile (\textit{Car}) è dunque vista come un insieme di sistemi (\textit{System}), i quali sono composti da vari tipi di componente (\textit{Component}).
Inoltre, macchine e componenti sono costruiti da un produttore (\textit{Manufacturer}). Infine, di una macchina si vuole anche modellare il tipo di carburante (\textit{Fuel}) e la sua classe di emissioni (\textit{EuroEmissionClass}).\\

\section{Classi ed Object Properties}
A fronte dell'analisi effettuata, è stata individuata la gerarchia di classi. Il nucleo principale di questa gerarchia è il seguente:

\begin{itemize}
    \item \textbf{Car}: rappresenta un'automobile nel contesto di carpedia.
    \item \textbf{System}: rappresenta un sistema che compone un'automobile. Esistono sistemi di vari tipi, di seguito alcuni esempi:
          \begin{itemize}
              \item \textbf{BrakingSystem}: rappresenta il sistema frenante di un'automobile.
              \item \textbf{Engine}: rappresenta il motore di un'automobile.
              \item \textbf{SteeringSystem}: rappresenta il sistema di sterzo di un'automobile.
          \end{itemize}
    \item \textbf{Component}: rappresenta un componente di un sistema di un'automobile. In particolare, i componenti possono essere:
          \begin{itemize}
              \item \textbf{Sensor}: rappresenta un sensore di un'automobile.
              \item \textbf{Gauge}: rappresenta un indicatore di un'automobile, vale a dire un'interfaccia human-readable di una particolare misura rilevata da un sensore.
              \item \textbf{MechanicalComponent}: rappresenta un componente meccanico di un sistema dell'automobile.
          \end{itemize}
\end{itemize}

Il legame concettuale che lega queste tre classi principali è stato modellato attraverso due \textbf{Object Properties}:

\begin{itemize}
    \item \textbf{hasSystem}: rappresenta il legame tra un'automobile e un sistema che la compone.
    \item \textbf{hasComponent}: rappresenta il legame tra un sistema e un componente che lo forma. Siccome esistono tre tipi di Component, questa object property possiede 3 sotto-proprietà:
          \begin{itemize}
              \item \textbf{hasSensor}: rappresenta il legame tra un sistema e un sensore che lo forma.
              \item \textbf{hasGauge}: rappresenta il legame tra un sistema e un indicatore che lo forma.
              \item \textbf{hasMechanicalComponent}: rappresenta il legame tra un sistema e un componente meccanico che lo forma.
          \end{itemize}
\end{itemize}

Queste object properties hanno anche una variante inversa, che permette di risalire la gerarchia. In particolare, si ha:

\begin{itemize}
    \item \textbf{systemOf}: rappresenta il legame tra un sistema e l'automobile di cui fa parte.
    \item \textbf{componentOf}: rappresenta il legame tra un componente e il sistema di cui fa parte. Le sue sotto-proprietà sono:
          \begin{itemize}
              \item \textbf{sensorOf}: rappresenta il legame tra un sensore e il sistema di cui fa parte.
              \item \textbf{gaugeOf}: rappresenta il legame tra un indicatore e il sistema di cui fa parte.
              \item \textbf{mechanicalComponentOf}: rappresenta il legame tra un componente meccanico e il sistema di cui fa parte.
          \end{itemize}
\end{itemize}

A questo core principale di concetti sono stati affiancati altri elementi ritenuti utili dal gruppo per modellare al meglio il dominio. In particolare, sono state aggiunte le seguenti classi:

\begin{itemize}
    \item \textbf{Manufacturer}: rappresenta un produttore di un'automobile o di un componente.
    \item \textbf{Fuel}: rappresenta il tipo di carburante utilizzato da un'automobile.
    \item \textbf{EuroEmissionClass}: rappresenta la classe di emissioni di un'automobile.
\end{itemize}

Queste classi sono legate al resto dell'ontologia attraverso le seguenti object properties:

\begin{itemize}
    \item \textbf{hasManufacturer}: rappresenta il legame tra un'automobile e il suo produttore.
    \item \textbf{hasFuel}: rappresenta il legame tra un'automobile e il suo tipo di carburante.
    \item \textbf{hasEuroEmissionClass}: rappresenta il legame tra un'automobile e la sua classe di emissioni.
\end{itemize}

\subsection{Data Properties}
Nella sezione precedente abbiamo descritto il nucleo principale di classi e object properties che compongono l'ontologia. Tuttavia, per modellare al meglio alcuni concetti di dominio ritenuti utili, sono state aggiunte anche delle \textbf{Data Properties}:\\

\begin{itemize}
    \item \textbf{Car data properties}: sono data properties legate alla classe Car, modellano alcuni aspetti inerenti all'automobile come:
          \begin{itemize}
              \item \textbf{hasKilometers}: rappresenta i kilometri percorsi dall'automobile, utile in ambito di ricerca di auto usate nel mercato secondario automobilistico.
              \item \textbf{hasMaxSpeedKmph}: rappresenta la velocità massima dell'automobile in kilometri orari.
              \item \textbf{hasPrice}: rappresenta il prezzo dell'automobile.
              \item \textbf{hasSeats}: rappresenta il numero di posti dell'automobile.
              \item \textbf{hasWeight}: rappresenta il peso dell'automobile.
              \item \textbf{hasModel}: rappresenta il modello dell'automobile.
          \end{itemize}
    \item \textbf{Component data properties}: sono data properties legate alla classe Component, modellano alcuni aspetti inerenti al componente come:
          \begin{itemize}
              \item \textbf{hasSerialNumber}: rappresenta il numero di serie del componente.
          \end{itemize}
    \item \textbf{Engine data properties}: sono data properties legate alla classe Engine, un importante tipo di sistema, modellano alcuni aspetti inerenti al motore come:
          \begin{itemize}
              \item \textbf{hasDisplacement}: rappresenta la cilindrata del motore.
              \item \textbf{hasKilowatts}: rappresenta la potenza del motore in kilowatt.
          \end{itemize}
\end{itemize}