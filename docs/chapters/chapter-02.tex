\chapter{Analisi dei requisiti}
Il macro dominio in cui si colloca l'ontologia Carpedia è quello dell'\textit{automobile}. In particolare, con automobile si intende un veicolo a motore, con quattro ruote,
progettato per il trasporto di persone e destinato ad uso privato del cittadino.
Saranno pertanto esclusi dall'analisi i concetti relativi a mezzi di trasporto pubblico, commerciali o di lavoro.

In questo enorme contesto, l'ontologia si focalizza sui concetti prettamente relativi alle \textit{componenti} di un'automobile, siano esse sensoristiche o meccaniche, e dei \textit{sistemi} che
esse compongono.

Di un'automobile si vogliono modellare i principali sistemi, come ad esempio:

\begin{itemize}
    \item sistema frenante;
    \item motore di un'automobile: di questo sistema si vogliono modellare anche cilindrata e potenza;
    \item sistema di sterzo;
\end{itemize}

Ogni sistema è composto da una serie di componenti, che possono essere di tre tipi:
\begin{itemize}
    \item sensore;
    \item indicatore: vale a dire un'interfaccia human-readable di una particolare misura rilevata da un sensore;
    \item componente meccanico di un sistema dell'automobile;
\end{itemize}

Dei componenti si vuole modellare anche il produttore ed il numero di serie.

Inoltre, si vuole anche modellare:

\begin{itemize}
    \item produttore di un'automobile o di un componente;
    \item il tipo di carburante utilizzato da un'automobile;
    \item la classe di emissioni di un'automobile;
\end{itemize}

Infine, si vogliono modellare ulteriori elementi dell'automobile, come ad esempio:

\begin{itemize}
    \item kilometri percorsi;
    \item velocità massima dell'automobile espressa in kilometri orari;
    \item prezzo;
    \item numero di posti;
    \item peso;
    \item modello dell'automobile.
\end{itemize}