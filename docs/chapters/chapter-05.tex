\chapter{Conclusioni}

In questo progetto, ci siamo dedicati allo sviluppo di un'ontologia finalizzata a rappresentare i componenti automobilistici e le relazioni che li legano. Carpedia si configura come uno strumento progettato per affrontare i casi d'uso delineati nei capitoli precedenti.

Riteniamo che Carpedia fornisca una base per l'esplorazione e la comprensione dell'ecosistema automobilistico. La sua struttura e la gerarchia delle classi di emissioni contribuiscono in modo significativo a modellare l'impatto ambientale e la conformità dei veicoli alle normative, riflettendo l'evoluzione del settore automobilistico.

La flessibilità dell'ontologia emerge nelle regole SWRL e nelle query SPARQL implementate, consentendo di derivare nuove informazioni e di eseguire interrogazioni significative. Prevediamo che questa flessibilità sia di vitale importanza per adattarsi a futuri sviluppi e per estendere l'ontologia in risposta alle nuove esigenze emergenti nel campo dell'industria automobilistica.

Il nostro lavoro si conclude qui, ma vediamo questo progetto come un punto di partenza per future evoluzioni. La possibilità di estendere e migliorare l'ontologia, di affrontare nuovi casi d'uso e di integrare ulteriori dettagli e relazioni riflette la natura dinamica e in continua evoluzione del settore automobilistico.
