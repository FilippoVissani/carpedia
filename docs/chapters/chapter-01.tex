\chapter{Introduzione}

\section{Contesto}
Nel campo dell'ingegneria automobilistica, la sintesi e l'organizzazione di una conoscenza estesa sono imperativi per guidare il progresso e l'innovazione. L'intricato intreccio di relazioni tra diverse entità all'interno dell'ecosistema automobilistico richiede un approccio strutturato e completo alla rappresentazione della conoscenza. Questo capitolo introduce Carpedia, un'ontologia progettata pronta per servire come fondamento per organizzare sistematicamente e rappresentare la conoscenza nel dominio dell'industria automobilistica.

\section{Ambito e Applicazioni}
Oltre a essere un quadro concettuale, le applicazioni dell'ontologia si estendono a vari contesti critici. Questo capitolo approfondisce scenari specifici in cui l'ontologia si dimostra essere una risorsa utile. In particolare, i casi d'uso individuati sono:

\begin{itemize}
    \item \textbf{Rappresentazione della conoscenza}: l'ontologia è stata progettata per rappresentare la conoscenza nel dominio automobilistico.
    \item \textbf{Integrazione tra applicativi in ambito automobilistico}: in modo che essi possano fare riferimento a una fonte comune di conoscenza.
    \item \textbf{Integrazione con servizi come online car marketing}
    \item \textbf{Integrazione con cataloghi di componenti automobilistici}
\end{itemize}
